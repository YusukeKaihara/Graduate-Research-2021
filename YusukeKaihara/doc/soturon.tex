\documentclass{thesis}

% TX Fonts を使う
\usepackage{txfonts}
\usepackage{graphicx}
\usepackage{comment}
\usepackage{url}
\usepackage{listings}
\begin{document}


% 目次
\tableofcontents

\chapter{研究の背景}

インターネットの普及に伴いマルウェアが急増しており,マルウェアによる被害が深刻化している.マルウェアの感染を防ぐには,アンチウイルスソフトウェアのインストールが不可欠である.しかし,マルウェアは高度化しており,アンチウイルスソフトウェアでは検出されないものもある.多くのマルウェアは,暗号化や難読化などの手法を使用することで,アンチウイルスソフトウェアによる検出を回避している.


また,現在のマルウェアの多くは,ツールによって自動的に生成された既存のマルウェアの亜種である.大量の亜種を効率的に解析するには,事前に機能を推定することが効果的である.また,マルウェアのすべての動作が解析されていなくても,動的解析の初期段階でマルウェアの機能を推定できることが望ましい.



マルウェアの機能推定に関する関連研究について,大久保らの研究 \cite{okubo} では,マルウェアの静的解析結果からバイトコードを抽出し,LCS や N-gram を用いて検体間の類似度を求め,検体の推定機能に対してポイントを付与し判別分析法により機能推定を行う方法を提案した.しかし,この手法では,機能推定の精度があまり高くないことと,機能推定に時間がかかるというという問題点があった.その問題点を受けて,
児玉らの研究\cite{pre1}では,動的解析結果から得られるAPI コー ル列と保有機能の関係を学習し機能推定を行う方法を提案した.
しかし,この手法では,動的解析では動作環境によって挙動が変わってくる場合があるためAPIコールが呼び出されない場合に対応できないという問題点がある.


そこで,本研究では,動的解析の段階で欠損したと考えられるAPIを補う手法を提案する.提案手法では,同一ファミリの検体は類似した機能を保有するという性質を考慮して,欠損した可能性が高いと考えられるAPIをランダムに補う機能推定の検討を行った.機能推定については,SVMを用いて API コール列の特徴と保有機能の関係を学習することにより推定器を作成した.各機能の推定結果から平均正解率 89.25\%,
平均再現率86.67\%, 平均適合率 87.36\%,F値87.01\%が得られた.従来手法と比較すると保有機能の検出を特にF値においてより高い値を得ることができた.
以下の章構成について述べる.第 2章では機械学習について述べる.第 3章ではマルウェアの解析手法と FFRI Dataset について述べる.第 4章では従来手法と提案手法について述べる.第 5 章では詳細な実験内容と実験結果について述べる.第 6章では結論を述べる.

\chapter{機械学習と評価手法}
本章ではマルウェアの機能推定を行う際に使用したSVM \cite{機械学習} のアルゴリズムおよび評価指標について述べる.


\section{機械学習(Machine Learning)}
機械学習(Machine Learning)は,電子メールのウイルス対策や写真の自動タグ付け,映画の推薦など身近なあらゆるところで利用されている.機械学習は,人工知能(Artifical Inteligence)の一種であり,コンピュータが過去のデータに基づいて未来を予測することを可能にする.また,機械学習のほとんどの問題は以下の3つの主なカテゴリのいずれかに属する.
\subsection*{教師あり学習}
教師あり学習では,それぞれのデータ点にカテゴリラベルや数値ラベルが付与されている.カテゴリラベルは,例えば,犬や猫を含む画像に対する,「犬」や「猫」などのようなラベルである.数値ラベルは,中古の車につけられた売値のようなものを指す.教師あり学習の目的は,大量のラベル付きのデータ例(学習データ)に基づいて新しく得られたデータのラベルを予測することでなる.データ点に付与されたラベルがカテゴリラベルの場合は,分類問題,数値ラベルの場合は,回帰問題と呼ばれる.
\subsection*{教師なし学習}
教師なし学習では,データ点はラベルを持たない.教師なし学習の目的は,何らかの方法で,データをまとめる,もしくは,データが持つ構造を見つけることである.
\subsection*{強化学習}
強化学習では,与えられたそれぞれのデータ点に対する動作を選択するアルゴリズムを学習する.これはロボティクス分野でよく用いられる手法で,ある時刻での各種センサからの出力をデータ点として,ロボットの次の動作を決定する場合などに利用される.また,Internet of Things(IoT)への応用も可能である.この場合アルゴリズムは少し未来の時点で動作選択の適切さを示す報酬信号を受け取り,より高い報酬信号を得るために動作選択の戦略を修正する.


\section{SVM (Support Vector Machine)}
SVMは,2クラス分類の線形識別関数を構築する機械学習モデルの一種である.本研究では分類を扱った.SVMの目的は,常に分類エラーを最小化することであり,あるクラスのほかのクラスに対するデータ点のマージン(決定境界とクラス端点間の距離)を最大化することで決定境界を引く.線形決定境界を用いてデータを適切に分離できない場合,元の特徴の非線形結合を作る.つまり,データが線形分離可能となるような,より高次元の空間にデータを写像(例えば,2次元から3次元へ)と等価である.そうして,高次元空間において線形決定境界(すなわち,3次元においては平面)を探索する.しかし,この写像アプローチは,次元の間で数学的な写像を行うために,多くの項を導入する必要があるため,次元が大きくなると実用的でない点が問題となる.その問題点を解決するための関数にカーネル関数を用いて計算する.カーネル関数の例に,放射基底関数(Radial Basis Function;RBF)やガウス関数(釣鐘曲線)がある.


\section{評価手法}
本研究で使用する評価指標について述べる.モデル性能を評価するための混同行列を表\ref{混同行列}に示す.混同行列は,機能$F$を保有する検体に対して,正しく保有していると予測した場合の数 (true positive),機能Fを保有する検体に対して,間違って保有していないと予測した場合の数 (false positive),保有していない検体に対して,正しく保有していないと予測した場合の数 (true negative),保有していない検体に対して,間違って保有していると予測した場合の数 (false negative) をまとめると,次のような 2 × 2 の行列にまとめることができる.

\begin{table}[H]
	\caption{混同行列}
      \label{混同行列}
	\begin{center}
		\begin{tabular}{|l|l|l|l|} \hline
			\multicolumn{2}{|c|}{} & \multicolumn{2}{|c|}{機能$F$の有無} \\ \cline{3-4}
			\multicolumn{2}{|c|}{} & Positive & Negative \\ \hline
			機械学習モデルの予測 & Positive & True Positive(TP) & False Positive(FP) \\ \cline{2-4}
			& Negative & False Negative(FN) & True Negative(TN) \\ \hline
		\end{tabular}
	\end{center}
\end{table}

機械学習のモデルの評価には正解率 $(Accuracy)$,再現率 $(Recall)$,適合率 (Precision),F値$(F-measure)$ の$4$種類を使用する.


正解率($Accuracy$)は式\ref{formula:accuracy}で定義されていて,機能F について推定した全ての検体のなかで,モデルの予測と機能$F$の有無が一致していた割合を表す.
\begin{equation}
	\label{formula:accuracy}
	Accuracy = \frac{TP + TN}{TP + FP + TN + FN}
\end{equation}


再現率$(Recall)$は,式\ref{formula:recall}で定義されていて,機能$F$を保有する検体のうち,モデルが保有していると判定した割合を表す.
% Recall
\begin{equation}
	\label{formula:recall}
	Recall = \frac{TP}{TP + FN}
\end{equation}

適合率$(Precision)$は,式\ref{formula:precision}で定義されていて,機能$F$を保有すると推定した検体のうち,実際に機能$F$を保有していたものの割合を表す.

% Precision
\begin{equation}
	\label{formula:precision}
	Precision = \frac{TP}{TP + FP}
\end{equation}

%F-measure
F値$(F-measure)$は,式\ref{formula:Fmeasure}で定義されていて,再現率と適合率の調和平均を表す.
\begin{equation}
	\label{formula:Fmeasure}
	F = \frac{2Recall \times Precision}{Recall+Precision}
\end{equation}


\chapter{マルウェアの解析手法とFFRI Dataset}
本章では,マルウェアの解析手法と,研究者向けにデータセットとして配布されているFFRIDataset について述べる.

\section{マルウェアの解析手法}
マルウェアの挙動を知るための解析手法は大きく分けて,表層解析,静的解析,動的解析の3つに分けられる\cite{解析}.

\subsection*{表層解析}
表層解析は,ファイル自体が悪性であるかどうかの情報収集や,ファイルのメタ情報の収集を目的として行う解析プロセスである.アンチウイルスソフトで判定を行い,既知のマルウェアであるかを判断や,ファイルタイプや動作するCPUアーキテクチャの情報を収集,特徴的なバイト列(通信先URLやマルウェアが用いるコマンド等)の収集を行う.表層解析では,解析対象に対してツールを適用して情報を取得する.その他の解析のように,実際にマルウェアを動作させたりマルウェア内のプログラムコードを分析はしない.

\subsection*{動的解析}
動的解析(ブラックボックス解析)は,マルウェアが実際に動作した際に端末上にどのような痕跡が残るのか,またどのような通信が発生するのかの情報収集を目的とした解析プロセスである.監視ツールをインストールした環境で実際にマルウェアを動作させ,ファイルやレジストリアクセスの情報の収集や,通信を監視して通信先のIPアドレスやドメイン,URL,通信ペイロードなどの情報を収集を行う.動的解析では,プログラムコードを詳細に分析しないためブラックボックス解析とも呼ばれる.

\subsection*{静的解析}
静的解析(ホワイトボックス解析)は,逆アセンブラやデバッガを用いてマルウェアのプログラムコードを分析し,具備されている機能や特徴的なバイト列など詳細な情報を収集することを目的とした解析プロセスである.動的解析で実行されなかったコードを分析し,潜在的に保有している機能を明らかにし,マルウェア独自の通信プロトコルや通信先生成アルゴリズムのような動的解析だけでは特定が難しい情報の収集を行う.静的解析では,プログラムコードを詳細に分析するためホワイトボックス解析とも呼ばれる.


解析の難易度は,表層解析,動的解析,静的解析の順に上がっていく.
そのため,解析対象の数という観点で見た場合,表層解析が最も多くのマルウェア解析することができ,静的解析では限られた数のマルウェアしか解析することができない.
表\ref{解析プロセス}に示すように,解析には長所と短所がある.各解析の長所・短所を理解し,解析の目的の応じて解析プロセスの選択・組み合わせを行うことが重要である.

\begin{table}[H]
\begin{center}
\caption{各解析プロセスの長所と短所}
\label{解析プロセス}
\footnotesize
\begin{tabular}{|p{20truemm}||p{42truemm}|p{42truemm}|} \hline
 & 長所 & 短所 \\ \hline 
表層解析 &
\begin{minipage}{42truemm} 
  \begin{itemize}
    \setlength{\leftskip}{-8truemm}
    \item 短時間に解析結果を取得できる
    \item 解析者に要求されるスキルレベルは高くない\\
  \end{itemize}
\end{minipage}
&
\begin{minipage}{42truemm} %ここで列の幅が決まります
  \begin{itemize}
    \setlength{\leftskip}{-8truemm}
    \item 得られる情報が限定的である
    \item 難読化されたマルウェアからは十分な解析結果が得られない\\
  \end{itemize}
\end{minipage}
\\ \hline
動的解析&
\begin{minipage}{42truemm} 
  \begin{itemize}
    \setlength{\leftskip}{-8truemm}
    \item 解析者に要求スキルレベルは高くない
    \item 難読化されたマルウェアからも解析結果を取得できる
    \item 短時間で解析結果を取得できる\\
  \end{itemize}
\end{minipage}
&
\begin{minipage}{42truemm} %ここで列の幅が決まります
  \begin{itemize}
    \setlength{\leftskip}{-8truemm}
    \item 安全な解析環境を構築する必要がある
    \item 解析妨害機能を有する検体を十分に解析できないことがある
   \item 解析時に実行されなかったコードの振る舞いはわからない\\
  \end{itemize}
\end{minipage}
\\ \hline
静的解析&
\begin{minipage}{42truemm} 
  \begin{itemize}
    \setlength{\leftskip}{-8truemm}
    \item 動的解析で実行されないコードの動作を把握できる
    \item 具備された機能の詳細なアルゴリズムを解明できる
\\
  \end{itemize}
\end{minipage}
&
\begin{minipage}{42truemm} %ここで列の幅が決まります
  \begin{itemize}
    \setlength{\leftskip}{-8truemm}
    \item 解析者に要求されるスキルレベルが高い
    \item 詳細な解析結果を取得するのに時間がかかる\\

  \end{itemize}

\end{minipage}
\\ \hline

\end{tabular}
\end{center}
\end{table}
\section{FFRI Dataset}

本研究では,情報処理学会コンピュータセキュリティ研究会マルウェア対策人材育成ワークショップ (MWS) が研究者向けに配布しているFFRI Datasetを使用する\cite{pre3}.現時点では,2013-2017では動的解析ログ,2018-2021では表層解析ログが提供されている.
本研究では,マルウェアの機能を推定するため,マルウェアの機能概要が出力されている FFRI Dataset 2016, 2017 を実験に使用する.


FFRI Dataset 2016は2016年の1月から2016年の3月までに収集された検体,計8,243検体分の動的解析ログである.これらの検体は,PE形式かつ実行可能なものであり,それぞれ,10ベンダー以上でマルウァア判定を受けている.仮想環境内でマルウェアを実行し,実行時のふるまいを90秒モニタリングしたログをjson形式で保存している.具体的なデータ項目は表\ref{table:2016}\cite{2016}である.

\begin{table}[H]
\begin{center}
  \caption{データ項目 2016}
\label{table:2016}
\begin{tabular}{|l|l|} \hline
項目 & 内容  \\ 
\hline \hline
info & 解析の開始,終了時刻,id等(idは1から順に採番) \\ \hline
signatures &ユーザー定義シグニチャとの照合結果(今回は使用無)  \\ \hline
virustotal &VirusTotalの検査履歴との照合結果(検体のMD5値に基づく)  \\ \hline
 static& 検体のファイル情報(インポートAPI,セクション構造等) \\ \hline
 dropped& 検体が実行時に生成したファイル \\ \hline
behavior &検体実行時のAPIログ(PID,TID,API名,引数,返り値等)  \\ \hline
processtree  &検体実行時のプロセスツリー(親子関係)  \\ \hline
summary &検体が実行時にアクセスしたファイル,レジストリ等の概要情報  \\ \hline
 target& 解析対象検体のファイル情報(ハッシュ値等) \\ \hline
debug & 検体解析時のCuckoo Sandboxのデバッグログ \\ \hline
strings  & 検体中に含まれる文字列情報 \\ \hline
network & 検体が実行時に行った通信の概要情報 \\ \hline
\end{tabular}  
\end{center}
\end{table}
FRI Dataset 2017は2017年の3月から2017年の4月までに収集された検体,計6,251検体分の動的解析ログである.これらの検体は,PE形式かつ実行可能なものであり,それぞれ,15ベンダー以上でマルウァア判定を受けている.仮想環境内でマルウェアを実行し,実行時のふるまいを90秒モニタリングしたログをjson形式で保存している.具体的なデータ項目は表\ref{table:2017}\cite{2017}である.
\begin{table}[H]
\begin{center}
  \caption{データ項目 2017}
\label{table:2017}
\begin{tabular}{|l|l|} \hline
項目 & 内容  \\ 
\hline \hline
info & 解析の開始,終了時刻,id等 \\ \hline
signatures &ユーザー定義シグニチャとの照合結果  \\ \hline
virustotal &   VirusTotalから得られる情報 \\ \hline
 static& 検体のファイル情報(インポートAPI,セクション構造等) \\ \hline
 dropped& 検体が実行時に生成したファイル \\ \hline
behavior &検体実行時のAPIログ(PID,TID,API名,引数,返り値等)  \\ \hline
 target& 解析対象検体のファイル情報(ハッシュ値等) \\ \hline
debug & 検体解析時のCuckoo Sandboxのデバッグログ \\ \hline
strings  & 検体中に含まれる文字列情報 \\ \hline
network & 検体が実行時に行った通信の概要情報 \\ \hline
\end{tabular}  
\end{center}
\end{table}


図\ref{listing:json}は,FFRI Datasetから取得できる,本実験で使用するファミリBackdoor.Win32.Andromのある検体の動的解析結果をjsonファイルに出力したものである.本研究では,マルウェアの動的解析結果から,マルウェアの実行時に呼び出されたWin32 API(以下,API)と動作概要を抽出する.ファミリ名については5行目の"Backdoor.Win32.Androm"といったファミリ名と定義する.

APIについては,21,27行目の "NtAllocateVirtualMemory", "LdrGetDllHandle"といったAPI関数名を抽出する.

動作概要については,35~38行目の"file\_created", "file\_recreated","directory\_created","dll\_loaded"といった動作概要名を抽出する.また,これらの動作概要名をこの検体が保有する機能として定義する.

\newpage

% Option of Listing Package
\lstset{
	basicstyle={\ttfamily},
	identifierstyle={\small},
	commentstyle={\smallitshape},
	keywordstyle={\small\bfseries},
	ndkeywordstyle={\small},
	stringstyle={\small\ttfamily},
	breakindent = 10pt,
	frame = trbl,
	tabsize = 4,
	lineskip = -0.5ex,
	captionpos = b,
	numbers = left
}
\renewcommand{\lstlistingname}{図}

% jsonファイルの内容
\begin{lstlisting}[caption={jsonファイルの内容}, label={listing:json}]
{
    "virustotal":{
              "scans" {
                  "Kaspersky":{
                         "result":"Backdoor.Win32.Androm.jjcl"
                         ...
                                    }.
                          }.
                        ...
                        }                        
	...
	"behavior": {
		"processes":[
			{...},
			{	
				"process_path": ... ,
				"calls": [
					{
						"category": "process",
						...
						"api": "NtAllocateVirtualMemory",
						...
					},
					{
						"category": "system",
						...
						"api": "LdrGetDllHandle",
						...
					},
					...
				], ...
			}
		], ...
		"summary": {
			"file_created": [...],
			"file_recreated": [...],
			"directory_created": [...],
			"dll_loaded": [...],
             ...
		}, ...
	}, 
	...
}
\end{lstlisting}











\chapter{APIコールの補完と機能推定の実験}
本章では,児玉らによって提案されたマルウェアの動的解析結果を用いた手法と,本研究で提案する手法について述べる.

\section{従来手法}
児玉らの手法ではマルウェアの動的解析結果から得られるAPI コー ル列と保有機能の関係を学習し機能推定するという手法が提案された.その手法は以下の通りである.


動的解析結果のAPIコール列を$C(c_1, c_2, \cdots, c_m)$とする.ここで,$c_t(t=1, 2, \cdots, m)$はAPI関数名を表す文字列とする.
APIコール列$C$に対して,API\ $c_t$の有無$e_{c_t}(C)$を
\[
	e_{c_t}(C) = \begin{cases}
		1 &有 \\
		0 & 無
	\end{cases}
\]
とした時,$m$次元特徴ベクトル$V(C)$を
\[
	V(C) = (e_{c_1}(C), e_{c_2}(C), \cdots, e_{c_m}(C)).
\]
と定義する.
$K$種類の推定機能$F_1, F_2, ...., F_K$とした時,推定する機能$F_k$に対する分類器は,動的解析結果のAPIコール列から得られる特徴ベクトル$V(C)$を入力として,推定機能$F_k$を保有する又は保有しないに分類する.

$N$個の動的解析結果$D_1,D_2,...,D_N$に対して,得られるAPIコール列を$C_1,C_2,...,C_N$とする.特徴ベクトル$V$を入力し,機能$F_i$を保有する又は保有しないを分類する分類器の作成方法は以下の手順となる.

\begin{enumerate}[(1)]
\item $D_l$に対して,機能$F_i$を保有する場合$L=1$,しない場合$L=0$とする.また,APIコール列$C_l$を抽出する.抽出したものを$(L,C_l)$で表現する.
\item $(L,C_l)$に対して,$C_l$を用いて3つの方法により特徴ベクトル$V(C_l)$を作成し,$(L, V(C_l))$を得る.
\item $(L, V(C_l))$をSVMの教師データとして学習させ,機能$F_i$を保有する又は保有しないを分類する分類器を得る.
\end{enumerate}

この分類器を利用して,機能が未定義のマルウェアの特徴ベクトル$V(C)$を与えたとき,$K$種類の各機能を保有する又は保有しないを推定するための手順は以下の通りである.

\begin{enumerate}
\item $K$個の分類器に特徴ベクトル$V(C)$を入力し,$(A_1, A_2, \cdots , A_K)$を得る.
\item $A=1$の時,機能を保有すると推定する.$A=0$の時,機能を保有しないと推定する.
\end{enumerate}
\section{従来手法の課題}
動的解析では動作環境によって挙動が変わってくる場合があるためAPIコールが呼び出されない場合がある図4.1は左図ではNtCriateFileが呼び出されているが,右図では呼び出されていない.従来手法ではこのようなAPIコールが欠損している状況に対応できないという課題がある.

\begin{figure}[htbp]
\begin{center}
  \begin{tabular}{c}

    % 1枚目の画像
    \begin{minipage}{0.5\hsize}
      \begin{center}
        \includegraphics[clip, width=60mm]{fig1.eps}

      \end{center}
    \end{minipage}

    % 2枚目の画像
    \begin{minipage}{0.5\hsize}
      \begin{center}
        \includegraphics[clip, width=60mm]{fig2.eps}
      \end{center}
    \end{minipage}

  \end{tabular}
  \caption{ ファミリBackdoor.Win32.Andromの2検体}
  \end{center}
\end{figure}


\section{提案手法}
本研究では欠損した可能性が高いAPIコールを補うことによって機能推定の精度の向上を図る手法を提案する.
動的解析結果のファミリ名を$g$とする.\\
g から n個の補完用の検体を取得する.$n$個の検体を
          $$ g_1, g_2, \cdots, g_n$$
と表す.また,検体$g_1$から$g_n$が呼び出す$k$種類のAPIコールを
               $$a_1, a_2, \cdots, a_k$$
と表す.また$g_1$から$g_n$が各APIコールを呼び出す回数を,
                $$b_1, b_2, \cdots, b_k$$
と表す.以上の情報を使って,以下の3つの手法を用いてAPIコールを補完する.\\
\begin{itemize}
\item[提案手法1] $a_1$から$a_k$から等確率でランダムにAPIを補完する.
\item[提案手法2]$B=b_1+b_2+\cdots+b_k $とすると,$b_i/B$の確率で$a_i $を補完する.例えば,APIコールの数$ k=4$ のとき,$b_1=1, b_2=1, b_3=2, b_4=3 $とすると,$a_1, a_2, a_3, a_4 $が補完される確率はそれぞれ$1/7, 1/7, 2/7, 3/7 $である.
\item[提案手法3]考え方は提案手法2と同様であるが.呼び出し回数が少ないAPIコールを優先して補完する.例えば,APIコールの数$ k=4$ のとき,$b_1=1, b_2=1, b_3=2, b_4=3$ とすると,$a_1, a_2, a_3, a_4$ が補完される確率はそれぞれ $3/7, 2/7, 1/7, 1/7$ である.
\end{itemize}
提案方法1~3に対し,従来手法と同様の方法で,特徴ベクトル$V_1$~$V_3$を作成する.
以降の分類器の作成,機能推定の手順は従来手法と同様である.

\section{検体の取得・推定機能の定義・データセットの作成}
マルウェアの機能推定をファミリごとに行うため,FFRI Dataset 2016, 2017 を「kaspersky社」命名のファミリ名ごとに分類した.実験の手法は提案手法に述べたとおりである.また,本実験で使用したファミリはBackdoor.Win32.Andromで,取得した検体数は$n=10$,実験用検体に付加するAPIの個数を$i=3$として特徴ベクトルを作成した.
取得した検体の呼び出すAPIから呼び出されている全ての API を取得すると,合計で $204$ 種類のAPIが得られた.
取得した全ての検体の動作概要から保有する機能を取得すると,合計で$ 26$ 種類の機能が
得られた.このことから,$26 $種類の機能を本研究で推定する機能として定義した.推定す
る機能の一覧を表 A.1 に示す.
ここで,呼び出される API 関数名は$ 204$ 種類であることから,特徴ベクトル $V_1$, $V_2$, $V_3$ の次元数は $204$ 次元としている.
また,本実験では不均衡データを取り扱うため,以下のimbalanced-learn が公開するアンダーサンプリングモジュールを使用した
\cite{undersampling}.
\[
	imblearn.under\_sampling.RandomUnderSampler
\]



\section{機械学習}
本研究では,データセットの特徴ベクトルを入力,ラベルを出力として機械学習モデルの
学習と評価を行う.ここで,データセットは推定する機能を保有する検体,保有しない検体
の特徴ベクトル,ラベルにより構成され,機械学習モデルは機能を保有する又は保有しない
を推定するため,2 値分類を行うモデルを推定する機能($26$ 種類)の数だけ用意する.

また,本研究では,$10$ 分割交差検証により分類器の検証
を行う.
$10$分割交差検証は,データセットを10個に分割してそのうち1つをテストデータに残りの$9$個を学習データとして正解率の評価を行う.
これを$10$個のデータすべてが$1$回ずつテストデータになるように$10$回学習を行なって精度の平均をとる手法である.
本研究で使用したSVMの実装では,scikit-learnが公開する分類問題を扱うSVMアルゴリズムである以下のモジュールを使用する\cite{svm}.
\[
	sklearn.svm.SVC
\]
このモジュールの全てのハイパーパラメータについてデフォルトの値を使用した($C=1.0, kernel='rbf', gamma='scale'$).

\section{実験結果と考察}
10分割交差検証を行い,各手法ごとに正解率,再現率,適合率,F値の平均値を算出した結果を表\ref{table:average}に示す.また,本実験では,精度の信頼性の観点からサンプル数が50以上の機能を対象とした.
\begin{table}[H]
	\caption{実験結果}
	\label{table:average}
	\begin{center}
		\begin{tabular}{|c||p{28mm}|p{28mm}|p{28mm}|p{28mm}|} \hline
			 & Accuracy Avg(\%) & Recall Avg(\%) & Precision Avg(\%) & F-measure \\ \hline \hline
			従来手法 & 89.26 & 86.62 &86.57 & 86.59  \\ \hline
			$V_1$ & 88.85  & 86.16 & 86.93 & 86.54 \\ \hline
			$V_2$ & 89.25 & 86.67 & 87.36 & 87.01 \\ \hline
			$V_3$ & 88.74 & 85.95 &86.94 & 86.44 \\ \hline
		\end{tabular}
	\end{center}
\end{table}
実験の結果から,正解率は従来手法が最も高いことと, 再現率,適合率,F値は$V_2$が最も高いことがわかる.モデルの評価において特に重要とされるF値の最も高い推定方法が$V_2$であることから,存在数の多いAPIを高確率で補完することが,機能推定に対して有効であることがわかる.
以上のことから従来手法と比べて提案手法が高いF値でマルウェアの機能推定を行えることがいえる.


従来手法とF値の最も高い推定方法である$V_2$の正解率,再現率,適合率,F値の平均値を各機能ごとに算出した結果を,表\ref{table:従来手法},表\ref{table:v2}に示す.

また,表\ref{table:従来手法},表\ref{table:v2}について,各評価値の単位を\%とする.
\begin{table}[H]
	\caption{従来手法}
	\label{table:従来手法}
	\begin{center}
		\begin{tabular}{|l||p{24mm}|p{24mm}|p{24mm}|p{24mm}|} \hline
			 & Accuracy Avg & Recall Avg & Precision Avg & F-measure \\ \hline \hline
			command\_line &94.37&93.35&90.29&91.79 \\ \hline
			connects\_ip &95.10&81.19&87.55&84.24 \\ \hline
			directory\_created &88.76&88.38&82.01&85.08 \\ \hline
			directory\_enumerated &86.79&82.84&93.60&87.89 \\ \hline
                  file\_copied  &94.86&88.90&95.46&92.07 \\ \hline
                   file\_created &82.88&87.38&74.54&80.45 \\ \hline
                   file\_deleted &85.79&71.89&76.89&74.31 \\ \hline
                  file\_exists&92.15&92.77&93.10&92.94 \\ \hline
                   file\_failed &87.51&85.27&96.66&90.61 \\ \hline
                  file\_read&96.33&96.72&97.29&97.00 \\ \hline
                  file\_recreated &83.33&80.97&87.46&84.09 \\ \hline
                  file\_written &84.55&91.58&75.23&82.93 \\ \hline
                   guid &84.84&90.81&78.09&79.43 \\ \hline
                  mutex &84.32&82.57&82.93&82.75 \\ \hline
                  regkey\_deleted&95.35&89.11&90.29&89.70 \\ \hline
                  regkey\_opened  &90.72&87.76&98.38&92.77 \\ \hline
                  regkey\_written &90.73&91.76&87.09&89.37 \\ \hline
                   resolves\_host &88.23&85.89&71.36&77.95 \\ \hline
		\end{tabular}
	\end{center}
\end{table}


\begin{table}[H]
	\caption{$v_2$}
	\label{table:v2}
	\begin{center}
		\begin{tabular}{|l||p{24mm}|p{24mm}|p{24mm}|p{24mm}|} \hline
			 & Accuracy Avg & Recall Avg & Precision Avg & F-measure \\ \hline \hline
			command\_line &  96.07&94.67  &94.20 &94.43  \\ \hline
			connects\_ip &94.50&79.52&86.87&83.03  \\ \hline
			directory\_created &87.15&86.43&80.88&83.56  \\ \hline
			directory\_enumerated &88.20&85.55&94.49&89.80 \\ \hline
                  file\_copied  &93.96&87.14&94.98&90.89 \\ \hline
                   file\_created &93.96&87.14&94.98&84.88 \\ \hline
                   file\_deleted &83.20&84.91&68.14&75.61 \\ \hline
                  file\_exists &92.65&93.01&94.21&93.61 \\ \hline
                   file\_failed &88.18&86.64&96.80&91.44 \\ \hline
                  file\_read &95.27&96.21&96.31&96.26 \\ \hline
                  file\_recreated &82.91&79.60&78.20&84.07 \\ \hline
                  file\_written &95.53&87.89&92.99&93.08 \\ \hline
                   guid &87.65&82.69&84.19&83.43 \\ \hline
                  mutex &81.38&79.60&78.20&78.89 \\ \hline
                  regkey\_deleted &95.53&87.8&92.99&90.37 \\ \hline
                  regkey\_opened  &90.02&88.82&96.79&92.63 \\ \hline
                  regkey\_written &89.25&85.29&89.46&87.33 \\ \hline
                   resolves\_host &88.21&87.22&73.29&79.65 \\ \hline
		\end{tabular}
	\end{center}
\end{table}





以上の結果から従来手法と$V_2$のF値を機能ごとに比較すると,
(command\_line,directory\_enumerated,\\ file\_created, 
file\_deleted, file\_exists,file\_failed,file\_written,guid, regkey\_deleted,resolves\_host)
の機能において$V_2$のF値が高いことがわかる.補完済みのAPIコール列が補完前のAPIコール列と比べ保有機能の特徴をとらえているため,補完に用いたAPIと保有機能が深く関係していると考えられる.
本実験で,補完に用いたファミリBackdoor.Win32.Andromの10検体が呼び出したAPIとその回数は表\ref{table:complete}である.
\begin{table}[H]
	\caption{補完に用いたAPIとその存在数}
	\label{table:complete}
	\begin{center}
		\begin{tabular}{|l|p{20mm}|} \hline 
			API & 存在数  \\ \hline\hline
			NtClose & 31 \\ \hline
			NtProtectVirtualMemory & 25 \\ \hline
			LdrGetDllHandle & 20 \\ \hline
			GetSystemMetrics & 14 \\ \hline
			NtOpenKeyEx &12  \\ \hline			
			NtAllocateVirtualMemory & 11 \\ \hline
			NtQueryValueKey & 10 \\ \hline
			NtMapViewOfSection &9  \\ \hline
			NtOpenSection & 8 \\ \hline
			NtOpenKey &7  \\ \hline
			FindResourceA &6  \\ \hline
			LdrUnloadDll &5  \\ \hline
			NtOpenFile &4  \\ \hline
                   SetWindowsHookExA &2  \\ \hline
                  NtFreeVirtualMemory &2  \\ \hline
                  NtCreateSection &2  \\ \hline
                   GetCursorPos &1  \\ \hline
                  RegCloseKey &1  \\ \hline
                  GetForegroundWindow  & 1 \\ \hline 
                  NtCreateFile  &1  \\ \hline
                  NtQueryAttributesFile  & 1 \\ \hline
                 NtCreateMutant &1  \\ \hline
                   CoInitializeEx &1  \\ \hline
                  OpenSCManagerA  &1  \\ \hline
                 EnumWindows&1   \\ \hline
		\end{tabular}
	\end{center}
\end{table}

\chapter{結論}
本研究では,同一ファミリから欠損した可能性が高い API コールを補い,SVMを用いて機械学習することにより,機能推定を行った.特徴ベクトル $V_2$ を用いた場合の提案手法の実験結果として,各機能の推定結果から平均正解率は約89.25\%,平均再現率は約86.67\%,平均適合率は約87.36\%,F値は約87.01\%,が得られた.従来手法と比較すると保有機能の検出を特にF値においてより高い値を得ることができた.


また,本研究ではファミリBackdoor.Win32.Andromのみを用いて実験を行ったが,ほかのファミリを用いての比較・検討を行うことが今後の課題である.

\acknowledgement
本研究を行うにあたり,常日頃より懇切丁寧に御指導いただきました高橋寛教授,甲斐 博 准教授,王森レイ講師に心より御礼申し上げます.また,本研究に際しご審
査頂きました遠藤 慶一准教授,宇戸 寿幸准教授に深く御礼申し上げます.
 最後に,日頃から助言や励ましをいただきました諸先輩方,並びに同研究室の皆様に深く御礼を申し上げます.

\begin{thebibliography}{99}


\bibitem{okubo}
大久保諒,伊沢亮一,森井昌克,井上大介,中尾康二: マルウェアの類似度に基づ
く機能推定,情報処理学会コンピュータセキュリティシンポジウム 2013(CSS2013),
pp.193-196, 2013.

\bibitem{pre1}
児玉光平:機械学習を用いたマルウェア機能推定に関する研究,愛媛大学修士論文,2020.

\bibitem{機械学習}
Michael Beyeler:Machine Learning for OpenCV Inteligent image processing with Python,2018.


\url{https://analysis-navi.com/?p=550}

\bibitem{解析}
八木毅,青木一史,秋山満昭,幾世知範,高田雄太,千葉大紀: 実践サイバーセキュリティモニタリング, コロナ社,2016.

\bibitem{pre3}
MWS Datasets.

\url{https://www.iwsec.org/mws/datasets.html}


\bibitem{2016}
FFRI:FFRI Dataset 2016のご紹介 - IWSEC .

\url{http://www.iwsec.org/mws/2016/20160530-ffri-dataset-2016.pdf}

\bibitem{2017}
FFRI:FFRI Dataset 2017のご紹介.

\url{https://www.iwsec.org/mws/2017/20170606/FFRI_Dataset_2017.pdf}

\bibitem{undersampling}
imblearn.under\_sampling.RandomUnderSampler.

\url{https://glemaitre.github.io/imbalanced-learn/generated/imblearn.under_sampling.RandomUnderSampler.html}

\bibitem{svm}
Support Vector Machines — scikit-learn 1.0.2 documentation.

\url{https://scikit-learn.org/stable/modules/generated/sklearn.svm.SVC.html}
\end{thebibliography}





\end{document}